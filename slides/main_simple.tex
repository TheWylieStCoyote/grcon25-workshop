\documentclass[aspectratio=169,11pt]{beamer}
\usetheme{Madrid}
\usecolortheme{seahorse}

% Basic packages
\usepackage{graphicx}
\usepackage{listings}
\usepackage{tikz}
\usepackage{amsmath}
\usepackage{hyperref}

% Title Information
\title{Introduction to GNU Radio}
\subtitle{Using GRC and Python for Software Defined Radio}
\author{Workshop Instructor}
\institute{GRCon 2025}
\date{\today}

\begin{document}

% Title slide
\begin{frame}
\titlepage
\end{frame}

% Table of contents
\begin{frame}{Agenda}
\tableofcontents
\end{frame}

\section{Introduction}
\begin{frame}{Welcome}
\begin{itemize}
    \item Welcome to the GNU Radio Workshop
    \item Today we'll learn about Software Defined Radio
    \item Using GNU Radio Companion (GRC)
    \item Python programming with GNU Radio
\end{itemize}
\end{frame}

\section{SDR Basics}
\begin{frame}{What is SDR?}
\begin{itemize}
    \item Software Defined Radio
    \item Process radio signals in software
    \item Flexible and reconfigurable
    \item Used for communications, research, education
\end{itemize}
\end{frame}

\section{GNU Radio Companion}
\begin{frame}{GRC Interface}
\begin{itemize}
    \item Visual programming environment
    \item Drag and drop blocks
    \item Connect signal flows
    \item Generate Python code
\end{itemize}
\end{frame}

\section{Conclusion}
\begin{frame}{Summary}
\begin{itemize}
    \item Learned SDR fundamentals
    \item Built flowgraphs in GRC
    \item Created Python applications
    \item Ready for advanced topics
\end{itemize}
\end{frame}

\begin{frame}{Questions?}
\centering
\Large Thank you for attending!\\[1em]
\normalsize
Contact: instructor@example.com
\end{frame}

\end{document}